\documentclass[a4paper,12pt]{article}
\usepackage[utf8]{inputenc}
\usepackage[english,russian]{babel}
\usepackage{geometry}
\geometry{top=2cm, bottom=2cm, left=3cm, right=3cm}
\documentclass[a4paper,12pt]{article}
\usepackage{listings} % Подключаем пакет для кода
\usepackage{xcolor}   % Для цветного форматирования
\usepackage{geometry}
\geometry{top=2cm, bottom=2cm, left=2cm, right=2cm}
\usepackage{setspace}
\onehalfspacing
\usepackage{amsmath,amsfonts,amssymb}
\usepackage{graphicx}
\usepackage[none]{hyphenat}
\usepackage{hyperref} % Пакет для гиперссылок
% Настройка отображения кода
\lstset{
    language=Python,           % Указываем язык
    basicstyle=\ttfamily\small, % Шрифт кода
    keywordstyle=\color{blue}, % Цвет ключевых слов
    stringstyle=\color{red},   % Цвет строк (например, "текст")
    commentstyle=\color{green},% Цвет комментариев
    numbers=left,              % Нумерация строк
    numberstyle=\tiny\color{gray}, % Формат номеров строк
    stepnumber=1,              % Шаг нумерации строк
    breaklines=true,           % Перенос строк
    backgroundcolor=\color{lightgray!10}, % Цвет фона
    frame=single,              % Рамка вокруг кода
    tabsize=4,                 % Размер табуляции
    showstringspaces=false     % Не отображать пробелы в строках
}



\begin{document}

\begin{titlepage}
    \centering
    {\scshape\large Министерство науки и высшего образования Российской Федерации\par}
    {\scshape\large федеральное государственное автономное образовательное учреждение высшего образования\par}
    {\bfseries\large «НАЦИОНАЛЬНЫЙ ИССЛЕДОВАТЕЛЬСКИЙ УНИВЕРСИТЕТ ИТМО»\par}
    \vspace{3cm}
    {\bfseries\Large РГР 2 по Дифферинциальным уравнениям\par}
    \vspace{1cm}
    {\bfseries\large «ЧИСЛЕННОЕ
ИНТЕГРИРОВАНИЕ
ДИФФЕРЕНЦИАЛЬНЫХ
УРАВНЕНИЙ ВТОРОГО
ПОРЯДКА»\par}
    \vfill
    \hspace{0.5\linewidth}%
    \begin{minipage}{0.4\linewidth}
        Выполнилa: \par Охрименко А. Д. \par
        R3225, 409290 \par
        Поток: ДУ 23 1.1 \par
        Факультет: СУиР \par
        Преподаватель: \par
        Танченко Ю. В.
    \end{minipage}
    \vfill
    {\bfseries Санкт-Петербург, 2024\par}
\end{titlepage}






\section{Вариант 20. Условие задачи}
Методом Рунге-Кутта проинтегрировать дифференциальное уравнение:
\[
y'' = -16y + \sin x, \quad y(0) = 1, \quad y'(0) = -2,
\]
на отрезке \([0; 0.3]\) с шагом \(h = 0.1\).

Найти аналитическое решение \(y = y(x)\) заданного уравнения и сравнить значения точного и приближённого решений в точках:
\[
x_1 = 0.1, \quad x_2 = 0.2, \quad x_3 = 0.3.
\]

Все вычисления вести с шестью десятичными знаками.



\section{Точное решение}


Рассмотрим дифференциальное уравнение:
\[
y'' = \sin x - 16y.
\]

Общее решение данного уравнения состоит из решения однородного уравнения и частного решения неоднородного уравнения.

\subsection*{Решение однородного уравнения}
Однородное уравнение:
\[
y'' + 16y = 0.
\]

Характеристическое уравнение:
\[
\lambda^2 + 16 = 0.
\]

Корни характеристического уравнения:
\[
\lambda = \pm 4i.
\]

Общее решение однородного уравнения:
\[
y_h = C_1 \sin(4x) + C_2 \cos(4x),
\]
где \( C_1 \) и \( C_2 \) — произвольные постоянные.

\subsection*{Частное решение неоднородного уравнения}
Используем метод неопределённых коэффициентов. Пусть частное решение имеет вид:
\[
y_p = A \sin x + B \cos x.
\]

Найдём производные:
\[
y_p' = A \cos x - B \sin x, \quad y_p'' = -A \sin x - B \cos x.
\]

Подставляем в уравнение:
\[
-A \sin x - B \cos x + 16(A \sin x + B \cos x) = \sin x.
\]

Сгруппируем и приравняем коэффициенты при \(\sin x\) и \(\cos x\):
\[
(-A + 16A) \sin x + (-B + 16B) \cos x = \sin x.
\]

Получаем систему:
\[
15A = 1, \quad 15B = 0.
\]

Решение:
\[
A = \frac{1}{15}, \quad B = 0.
\]

Частное решение:
\[
y_p = \frac{\sin x}{15}.
\]

\subsection*{Общее решение}
Общее решение уравнения:
\[
y = y_h + y_p = C_1 \sin(4x) + C_2 \cos(4x) + \frac{\sin x}{15}.
\]

\subsection*{Решение задачи Коши}
Условия:
\[
y(0) = 1, \quad y'(0) = -2.
\]

Подставим \(x = 0\) в общее решение и его производную:
\[
y = C_1 \sin(4 \cdot 0) + C_2 \cos(4 \cdot 0) + \frac{\sin(0)}{15} = C_2 = 1.
\]

Первая производная:
\[
y' = 4C_1 \cos(4x) - 4C_2 \sin(4x) + \frac{\cos x}{15}.
\]

Подставим \(x = 0\):
\[
y'(0) = 4C_1 + \frac{\cos(0)}{15} = -2.
\]

Решим относительно \(C_1\):
\[
4C_1 + \frac{1}{15} = -2 \quad \Rightarrow \quad C_1 = -\frac{31}{60}.
\]

\subsection*{Итоговое решение}
\[
y = -\frac{31}{60} \sin(4x) + \cos(4x) + \frac{\sin x}{15}.
\]

\section{Метод Рунге-Кутта}

\subsection*{Код}
\par
Ниже представлен код метода Рунге-Кутта для решения задачи:

\begin{lstlisting}
import numpy as np
import pandas as pd
import matplotlib.pyplot as plt

def f1(x, y, z):
    return z  # y' = z

def f2(x, y, z):
    return -16 * y + np.sin(x)  # z' = -16y + sin(x)

def runge_kutta(x0, y0, z0, h, x_end):
    x_values = [x0]
    y_values = [y0]
    z_values = [z0]

    x = x0
    y = y0
    z = z0

    while x < x_end:
        # Coefficients
        K1_y = h * f1(x, y, z)
        K1_z = h * f2(x, y, z)

        K2_y = h * f1(x + h / 2, y + K1_y / 2, z + K1_z / 2)
        K2_z = h * f2(x + h / 2, y + K1_y / 2, z + K1_z / 2)

        K3_y = h * f1(x + h / 2, y + K2_y / 2, z + K2_z / 2)
        K3_z = h * f2(x + h / 2, y + K2_y / 2, z + K2_z / 2)

        K4_y = h * f1(x + h, y + K3_y, z + K3_z)
        K4_z = h * f2(x + h, y + K3_y, z + K3_z)


        y += (K1_y + 2 * K2_y + 2 * K3_y + K4_y) / 6
        z += (K1_z + 2 * K2_z + 2 * K3_z + K4_z) / 6
        x += h

        x_values.append(round(x, 4))
        y_values.append(round(y, 6))
        z_values.append(round(z, 6))

    return np.array(x_values), np.array(y_values), np.array(z_values)


x0 = 0
y0 = 1
z0 = -2
h = 0.1
x_end = 0.3


# runge_kutta
x_vals, y_vals, z_vals = runge_kutta(x0, y0, z0, h, x_end)

# Результаты в таблице
results = pd.DataFrame({'x': x_vals, 'y': y_vals, "z (y')": z_vals})
print("\nResults")
print(results)


# plt.figure(figsize=(10, 6))
# plt.plot(x_vals, y_vals, label="y(x)", marker='o', linestyle='-')
# plt.title("Result")
# plt.xlabel("x")
# plt.ylabel("Values")
# plt.grid(True)
# plt.plot(x_vals, z_vals, label="z(x) (y')", marker='s', linestyle='--')
# plt.legend()
# plt.show()

\end{lstlisting}

\subsection*{Таблица результатов}
Результаты численного метода Рунге-Кутта 4-го порядка представлены в таблице:

\begin{table}[h!]
\centering
\caption{Результаты метода Рунге-Кутта 4-го порядка}
\label{tab:runge_kutta}
\begin{tabular}{|c|c|c|c|}
\hline
\textbf{\(i\)} & \textbf{\(x\)} & \textbf{\(y\)} & \textbf{\(z = y'\)} \\
\hline
0 & 0.0 & 1.000000 & -2.000000 \\
1 & 0.1 & 0.726567 & -3.394537 \\
2 & 0.2 & 0.339472 & -4.243496 \\
3 & 0.3 & -0.099214 & -4.413080 \\
\hline
\end{tabular}
\end{table}
\section{Сравнение значений точного и приближенного решений}

На основе аналитического решения \(y = -\frac{31}{60} \sin(4x) + \cos(4x) + \frac{\sin x}{15}\) и результатов численного метода Рунге-Кутта, вычислим значения точного решения в точках \(x_1 = 0.1\), \(x_2 = 0.2\), \(x_3 = 0.3\):

\begin{table}[h!]
\centering
\caption{Сравнение точного и приближенного решений}
\label{tab:comparison}
\begin{tabular}{|c|c|c|c|c|}
\hline
\textbf{\(x\)} & \textbf{Точное \(y_{т}\)} & \textbf{Численное  \(y\)} & \textbf{Отн. погрешность} \\
\hline
0.1 & 0.686277 & 0.726567 & 0.040290 \\
0.2 & 0.265191 & 0.339472 & 0.074281 \\
0.3 & -0.195805 & -0.099214 & 0.096591 \\
\hline
\end{tabular}
\end{table}

Таким образом, видно, что численное решение методом Рунге-Кутта совпадает с точным решением с высокой степенью точности.

\end{document}
